\section{Grundbegriffe der Wahrscheinlichkeitsrechnung}

Viele Vorgänge in unserer Welt können wir nicht präzise auf ihre Ursachen zurückführen - sie erscheinen uns "`zufällig"'. Wir versuchen dennoch dieses Verhalten mathematisch exakt zu beschreiben. 
%
%
%
\defi 
Ein Zufallsversuch sei ein Experiment dessen mögliche Ausgänge als \underline{Ergebnisse} bezeichnet werden. Die Menge aller Ergebnisse sei die Ergebnismenge $\Omega$ des Versuchs.
%
%
%
\bsp
\begin{description}
	\item[$\blacktriangleright$] Werfen von zwei Würfeln $\Omega = S \times S$ wobei 
			$S=\{1,\cdots,6\}$
	\item[$\blacktriangleright$] Die täglich von 12 Uhr an einer Tafel gemessene Temperatur
			ist ein Ergebnis aus $\Omega = \mathbb{R}$
\end{description}
%
%
%
\defi 
Jede Teilmenge der Ergebnismenge $\Omega$ bezeichnen wir als ein \underline{Ergebnis}.\\
Wir sagen: Das Ereignis $A \subseteq \Omega$ \underline{tritt ein}, falls eine Durchführung des Zufallsversuchs ein Ergebnis liefert, das in $A$ liegt. 
%
%
%
\bsp
\begin{description}
	\item[$\blacktriangleright$] Werfen zweier Würfel:\\
			Pasch =$ \{ (1,1),(2,2),(3,3),\dotsc,(6,6)\} \subseteq \Omega = S \times S$ wie 
			oben.
	\item[$\blacktriangleright$] Hörsaaltemperatur: Angenehm = $ [22,24] \subseteq 
			\mathbb{R}$
\end{description}
%
%
%
\defi
Eine \underline{Ergebnisalgebra} in $\Omega$ sei eine Familie von Ereignissen mit: \begin{description}
	\item[$\blacktriangleright$] $\Omega \in$ \suet{A}
	\item[$\blacktriangleright$] \normalfont{Ist }$A\subset$ \suet{A} \normalfont{, so 
			auch }$\Omega \diagdown A \in$ \suet{A} \normalfont{}
	\item[$\blacktriangleright$] sind $A_{n} \in $\, \suet{A}\normalfont{} 		
			$(n\in\mathbb{N})$ so auch 
			$\mathop{\bigcup}\limits_{n\in\mathbb{N}}A_{n} \in$ \,\suet{A}
			\normalfont{}
			
\end{description}

Eine \underline{Wahrscheinlichkeitsfunktion} sei eine Abbildung $p$\, \suet{A}\normalfont{} $\rightarrow$ [0,1] mit:
\begin{description}
	\item[$\blacktriangleright$] $p(\Omega)=1$
	\item[$\blacktriangleright$] Sind $A_{n} \in$ \suet{A}\normalfont{}
		$(n\in\mathbb{N})$ mit $A_{n} \cap A_{m} =$ \O
\end{description}

Wir nennen $(\Omega,p)$ einen \underline{Wahrscheinlichkeitsraum}

%
%
%

\bsp 
\begin{description}
	\item[$\blacktriangleright$] Ist $\Omega$ endlich, so wähle \suet{A}\normalfont{} = 
		$\gamma ( \Omega)$= Menge aller Teilmengen von $\Omega$ und 
		$p(A)=\sum\limits_{w\in A} p(w)$ für $A \subseteq \Omega$ sofern $p=\Omega 
		\rightarrow$ [1,0] mit $\sum\limits_{w\in\Omega}p(w) = 1$
	\item[$\blacktriangleright$] $\Omega$ = Alphabet der deutschen Sprache \\
		$p(w)$ = Wahrscheinlichkeit für das Auftreten der Buchstaben $w$ in einem 
		Text der deutschen Sprache.
	\item[$\blacktriangleright$] \quad Fairer Münzwurf:\\
		$\Omega = \{z,k\} \qquad p(z) = \frac{1}{2} = p(k)$\\
		Gezinkter Münzwurf:\\
		$\Omega = \{z,k\} \qquad p(z) \neq \frac{1}{2} \neq p(k)$\\
		wobei: $p(z) = p$ und $p(k) = 1-p$
	\item[$\blacktriangleright$] \quad Werfen zweier fairer Würfel:\\
		$\Omega = S\times S \qquad p(w) = \frac{1}{36}$ für alle $w \in \Omega$ wie 
		oben.
\end{description}	

%
%
%
\defi
Ist $|\Omega| < \infty$ und $p(w) = \frac{1}{|\Omega|}$ für alle $w \in \Omega$, so nennen wir $p$ \underline{Gleichverteilung} auf $\Omega$

%
%
%

\stz
Für Ereignisse $A,B \in $ \suet{A}\normalfont{} gilt stets:
\begin{enumerate}[a) ]
	\item $A\cap B \in$ \suet{A}\normalfont{}
	\item $p(\Omega \diagdown A)=1-p(A)$, insbesondere $p($\O$) = 0$
	\item ist $A \subseteq B$, so ist $p(A) \leq p(B)$
	\item $p(A\cup B) = p(A) + p(B) - p(A\cap B)$
\end{enumerate}
\qquad \\
\qquad\\
%
%
%

\bws \\
\begin{enumerate}[a) ]
	\item $\Omega\diagdown(A\cap B) \mathop{=}\limits^{\textrm{DE MORGAN}} 
		(\Omega \diagdown A) \cup (\Omega \diagdown B)$
	\item $\Omega = A \cup ( \Omega \diagdown A)$ \\
		$1=p(\Omega) = p(A) + p(\Omega \diagdown A)$
	\item $B=A\dot{\cup}(B\diagdown A)$ mit $B\diagdown A = B \cap (\Omega 
		\diagdown A) \in$ \suet{A} \normalfont{}
	\item $(A\cup B)\diagdown A = B \diagdown (A\cap B)$\\
		Ansonsten gab es nur Fehler bei dem erläuterungsversuch in dieser Vorlesung.
\end{enumerate}