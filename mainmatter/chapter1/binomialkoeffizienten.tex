\section{Binomialkoeffizienten}

%
%
%

\defi
\subsubsection{Fakultät}
$n! = n(n-1)(n-2)\dots1$\\
für $1\leq n\in\mathbb{N}$\\
$0!=1$\\

%
%
%

\stz
$n!$ ist die Anzahl der Möglichkeiten, $n$ Objekte anzuordnen (in einer Reihe)\\

%
%
%

\bws 

 \begin{tabular}{ll}
$n$  & Möglichkeiten für 1. Objekt \\
$(n-1)\dots$ & Möglichkeiten für 2. Objekt \\
$(n-2)\dots$ & Möglichkeiten für 3. Objekt\\
 \end{tabular}$\square$

%
%
%

\defi 
Es sei $\mathop{\underline{n}}\limits^{\mathop{\textrm{Standardmenge}}\limits_{\downarrow}} = \{0,1,\dots,n-1\}$\\ 
Für $0\leq k \leq n$ sie $\begin{pmatrix} n \\ k \end{pmatrix}$ die Anzahl  aller $k$-elementigen Teilmengen von $\underline{n}$. 

%
%
%

\stz
Für $0\leq k \leq n$ ist $\begin{pmatrix} n \\ k \end{pmatrix} = \frac{n!}{k! \cdot (n-k)!} $ \\
\bws \\
Wir basteln eine $k$-elemtige Teilmenge von $\underline{n}$ \\
\begin{tabular}{cc}
$n$ & Möglichkeiten für 1. Element der Teilmenge \\
$n-1$ & Möglichkeiten für 2. Element der Teilmenge  \\
 \vdots &  \vdots  \\
$n-k+1$ & Möglichkeiten für k. Element der Teilmenge \\
\end{tabular}
\\
Insgesamt $= n(n-1) \dotsc (n-k+1)=\frac{n!}{(n-k)!}$ Möglichkeiten. Jede $k$-elementige Teilmenge wurde $k!$-mal konstruiert, da die Reihenfolge des gewählten Elemente keine Relevanz hat.\\
\underline{Fazit:} es gibt $\frac{1}{k!} \cdot \frac{n!}{(n-k)!} \quad k$-elementige Teilmengen in $\underline{n}$

%
%
%

\bsp 
\begin{enumerate}[a) ]
	\item Anzahl der Partien bei einem Turnier "`jeder gegen jeden"'.\\
		$\begin{pmatrix} n \\ 2 \end{pmatrix} = \frac{n(n-1)}{1\cdot 2} = \frac{n(n-1)}{2}$
	\item Anzahl der Tips beim Lotto "'6 aus 49"`\\ 
		\begin{align}	
			\begin{pmatrix} 49 \\ 6 \end{pmatrix} &= \frac{49 \cdot \textcolor{pviolet}{48} 
			\cdot 47 \cdot 46 \cdot \textcolor{pblue}{45} \cdot 44}{1 \cdot 
			\textcolor{pviolet}{2} \cdot  \textcolor{pblue}{3} \cdot 
			\textcolor{pviolet}{4} \cdot \textcolor{pblue}{5} \cdot 
			\textcolor{pviolet}{6}} \\
			&= 49 \cdot 47 \cdot 46 \cdot 3 \cdot 44 \\
			&= 13.983.816
		\end{align}
\end{enumerate}