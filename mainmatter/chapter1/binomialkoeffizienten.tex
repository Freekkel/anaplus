\section{Binomialkoeffizienten}

%
%
%

\defi
\subsubsection{Fakultät}
$n! = n(n-1)(n-2)\dots1$\\
für $1\leq n\in\mathbb{N}$\\
$0!=1$\\

%
%
%

\stz
$n!$ ist die Anzahl der Möglichkeiten, $n$ Objekte anzuordnen (in einer Reihe)\\

%
%
%

\bws 

 \begin{tabular}{ll}
$n$  & Möglichkeiten für 1. Objekt \\
$(n-1)\dots$ & Möglichkeiten für 2. Objekt \\
$(n-2)\dots$ & Möglichkeiten für 3. Objekt\\
 \end{tabular}$\square$

%
%
%

\defi 
Es sei $\mathop{\underline{n}}\limits^{\mathop{\textrm{Standardmenge}}\limits_{\downarrow}} = \{0,1,\dots,n-1\}$\\ 
Für $0\leq k \leq n$ sie $\begin{pmatrix} n \\ k \end{pmatrix}$ die Anzahl  aller $k$-elementigen Teilmengen von \n. 

%
%
%

\stz
Für $0\leq k \leq n$ ist $\begin{pmatrix} n \\ k \end{pmatrix} = \frac{n!}{k! \cdot (n-k)!} $ \\
\bws \\
Wir basteln eine $k$-elemtige Teilmenge von \n \\
\begin{tabular}{cc}
$n$ & Möglichkeiten für 1. Element der Teilmenge \\
$n-1$ & Möglichkeiten für 2. Element der Teilmenge  \\
 \vdots &  \vdots  \\
$n-k+1$ & Möglichkeiten für k. Element der Teilmenge \\
\end{tabular}
\\
Insgesamt $= n(n-1) \dotsc (n-k+1)=\frac{n!}{(n-k)!}$ Möglichkeiten. Jede $k$-elementige Teilmenge wurde $k!$-mal konstruiert, da die Reihenfolge des gewählten Elemente keine Relevanz hat.\\
\underline{Fazit:} es gibt $\frac{1}{k!} \cdot \frac{n!}{(n-k)!} \quad k$-elementige Teilmengen in \n

%
%
%

\bsp 
\begin{enumerate}[a) ]
	\item Anzahl der Partien bei einem Turnier "`jeder gegen jeden"'.\\
		$\begin{pmatrix} n \\ 2 \end{pmatrix} = \frac{n(n-1)}{1\cdot 2} = \frac{n(n-1)}{2}$
	\item Anzahl der Tips beim Lotto "`6 aus 49"'\\ 
		\begin{align}	
			\begin{pmatrix} 49 \\ 6 \end{pmatrix} &= \frac{49 \cdot \textcolor{pviolet}{48} 
			\cdot 47 \cdot 46 \cdot \textcolor{pblue}{45} \cdot 44}{1 \cdot 
			\textcolor{pviolet}{2} \cdot  \textcolor{pblue}{3} \cdot 
			\textcolor{pviolet}{4} \cdot \textcolor{pblue}{5} \cdot 
			\textcolor{pviolet}{6}} \\
			&= 49 \cdot 47 \cdot 46 \cdot 3 \cdot 44 \\
			&= 13.983.816
		\end{align}
\end{enumerate}


%
%
%

\stz
Stets gilt: 
\begin{enumerate}[a) ]
	\item $\begin{pmatrix} n \\ 0 \end{pmatrix} = 1$ und $ \begin{pmatrix} n \\ n-k \end{pmatrix} = \begin{pmatrix} n \\ k \end{pmatrix}$ für $0  \leq k \leq n$
	\item $\begin{pmatrix} n \\ k \end{pmatrix} = \begin{pmatrix} n-1 \\ k-1 \end{pmatrix} + \begin{pmatrix} n-1 \\ k \end{pmatrix}$ für $ 1 \leq k \leq n-1$
	\item $\sum\limits^{n}_{k=0}  \begin{pmatrix} n \\ k \end{pmatrix} = 2^{n}$
\end{enumerate}

\bws
\begin{enumerate}[a) ]
	\item trivial
	\item Sei $M$ eine $k$-elementige Teilmenge von \n$=\{0,1,\dotsc,n-1\}$
	\begin{description}
		\item[\underline{1. Fall} $n-1\in M:$] $M=\{n-1\} \quad \dot{\bigcup} \quad M_{0}$ $k$-elementige Teilmenge von $\underline{n-1}$\\
			$\Rightarrow \begin{pmatrix} n-1 \\ k-1 \end{pmatrix}$ Möglichkeiten für $M_{0}$ bzw. $M$
		\item[\underline{2. Fall} $n-1 \notin M:$] $\mathop{M}\limits_{\mathop{k\textrm{-elementige Teilmenge}}\limits^{\uparrow}} \subseteq n-1$ \\
			$\Rightarrow \begin{pmatrix} n-1 \\ k \end{pmatrix}$ Möglichkeiten.
	\end{description}
	\item $\sum\limits^{n}_{k=0}\begin{pmatrix} n \\ k \end{pmatrix} =$ Anzahl aller Teilmengen von \n  \, $= 2^{n}$
\end{enumerate}

\begin{multicols}{2}
\underline{\large{PASCALsches Dreieck}}\\
\qquad\\
{\psset{nodesep=2pt, linewidth=0.5pt,rowsep=.2cm, colsep=0.25cm}%
\begin{psmatrix}
&          &  &  &  &  & 1&  &  &  &  &  &    n=0\\
&          &  &  &  & 1&  & 1&  &  &  &  &  n=1\\
&          &  &  & 1&  & 2&  &1 &  &  &  &n=2& & & & & &  \Huge{$\Rightarrow$}\\
&          &  & 1&  & 3&  & 3&  & 1&  &  & n=3\\
&          & 1&  & 4&  & 6&  & 4&  & 1&  &  n=4\\
&        1 & &5  & & 10 & &10  & & 5 & & 1 & n=5\\
\end{psmatrix}
}%

\small{}% Damit das spezielle Dreieck kleiner ist. 
{\psset{nodesep=1pt, linewidth=0.25pt,rowsep=.1cm, colsep=0.12cm}%
\begin{psmatrix}
         &  &  &  &  & $\begin{pmatrix} 0 \\ 0 \end{pmatrix}$&  &  &  &  &  &    n=0\\
          &  &  &  & $\begin{pmatrix}1\\0\end{pmatrix}$ &  & $\begin{pmatrix}1 \\ 1 \end{pmatrix}$ &  &  & & & n=1\\
          &  &  & $\begin{pmatrix} 2 \\ 0 \end{pmatrix} $&  & $\begin{pmatrix} 2 \\ 1 \end{pmatrix}$&  & $ \begin{pmatrix} 2 \\ 2 \end{pmatrix}$ &  &  &  &  n=2\\
          &  & $\begin{pmatrix} 3 \\ 0 \end{pmatrix}$ &  & $\begin{pmatrix} 3 \\ 1 \end{pmatrix}$&  & $\begin{pmatrix} 3 \\ 2 \end{pmatrix}$&  & $\begin{pmatrix} 3 \\3 \end{pmatrix}$ &  &  &  n=3\\
          & $\begin{pmatrix} 4 \\ 0 \end{pmatrix}$ &  & $\begin{pmatrix}4 \\ 1 \end{pmatrix} $&  & $\begin{pmatrix} 4 \\ 2 \end{pmatrix} $&  & $\begin{pmatrix} 4 \\ 3 \end{pmatrix}$&  & $\begin{pmatrix}4 \\ 4 \end{pmatrix}$&  &  n=4\\
         $\begin{pmatrix} 5 \\ 0 \end{pmatrix}$  & &$\begin{pmatrix}5\\ 1 \end{pmatrix}$  & & $\begin{pmatrix} 5\\2 \end{pmatrix}$ & &$\begin{pmatrix} 5 \\3 \end{pmatrix} $  & & $\begin{pmatrix}5\\ 4 \end{pmatrix}$ & & $\begin{pmatrix} 5 \\ 5 \end{pmatrix}$ & n=5\\
\end{psmatrix}
}%
\end{multicols}
\normalsize {} % damit wieder alles normalgroß ist

%
%
%

\subsection{Binomische Formel}
Für alle $x,y \in \mathbb{R}$ und $n \in \mathbb{N}$ ist $(x+y)=\sum\limits_{k=0}^{n}\begin{pmatrix} n \\ k \end{pmatrix}x^{k}y^{n-k}$
\qquad\\
\bws\\
$\mathop{\underbrace{(x+y)(x+y)\dotsc(x+y)}}\limits_{n-mal}$ aus multiplizieren liefert viele $x^{k}y^{n-k}$, nämlich so oft wie man $k$ der $^{n}$Klammern wählen kann, um daraus das $x$ zu rekrutieren $\Rightarrow \begin{pmatrix} n \\ k \end{pmatrix}$ Möglichkeiten