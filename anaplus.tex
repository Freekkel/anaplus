\documentclass[a4paper,10pt]{scrbook}
\usepackage{etex}
\usepackage[T1]{fontenc}
\usepackage[utf8x]{inputenc}
\usepackage[ngerman]{babel}
\usepackage{color}		%um farben zu definieren
\usepackage{multicol} %um temporär mehre spalten zu erzeugen
\usepackage{graphicx}
\usepackage{amssymb}
\usepackage{amsmath}
\usepackage{float}
\usepackage{latexsym}
\usepackage{pstricks}
\usepackage{stmaryrd}
\usepackage{MnSymbol}
\usepackage{array} % um schöne Formeln gestalten zu können.
\usepackage{linearb}
\usepackage{titlesec}
\usepackage{longtable} % für große Tabellen
\usepackage{enumerate} % für individuelles Weiterählen
\usepackage{pstricks-add} % Für das Pascalsche Dreieck
\usepackage{hyperref} % als letztes Packet einbinden zwingend notwendig


%Subsections werden hiermit nicht in das Inhaltsverzeichnis übernommen
\setcounter{tocdepth}{1}


%Mehr Raum zwischen subsections
\titlespacing{\section}{0pt}{*4}{*2.5}
\titlespacing{\subsection}{0pt}{*3.5}{*1.5}

% kleine Anpassungen, damit die Seitenbreite überall außer Titel gleich ist.
\setcounter{secnumdepth}{2}
\setlength{\textwidth}{160mm}
\setlength{\textheight}{220mm}
\setlength{\headheight}{3mm}
\evensidemargin1mm
\oddsidemargin1mm

%selbstdefinierte Befehle für schnelleres arbeiten
\newcommand{\tab}{\hspace*{3em}}
\newcommand{\RM}[1]{\MakeUppercase{\romannumeral #1}}
\newcommand{\bsp}{\subsection{\underline{Beispiel:}}}
\newcommand{\defi}{\subsection{\underline{Definition:}}}
\newcommand{\stz}{\subsection{\underline{Satz:}}}
\newcommand{\id}{\subsection{\underline{Idee:}}}
\newcommand{\bws}{\underline{Beweis:\\}}
\newcommand{\n}{$\underline{n}$}	%Standardmenge in text
\newcommand{\mn}{\underline{n}}	%Standardmenge in Mathe Umgebung


%geänderte befehle um den Tafelanschriften von Herr Leinen näher zu kommen.
\renewcommand{\thechapter}{\Roman{chapter}}
\renewcommand{\thesection}{\thechapter.\arabic{section}}
\renewcommand{\thesubsection}{\thesection.\arabic{subsection}}
\renewcommand{\thesubsubsection}{\thesubsection.\arabic{subsubsection}}

% Farbanpassung
\newrgbcolor{pgreen}{0 1 0}
\newrgbcolor{pblue}{0 0 1}
\newrgbcolor{plightgrey}{0.3 0.3 0.3}
\newrgbcolor{pgrey}{0.6 0.6 0.6}
\newrgbcolor{pdarkgrey}{0.8 0.8 0.8}
\newrgbcolor{pred}{1 0 0}
\newrgbcolor{pviolet}{1 0 1}

%Fonts für altdeutsche Buchstaben in Text einfügbar mit \suet{_TEXT_}

\newfont{\suet}{suet14}
\DeclareTextFontCommand{\textsuet}{\suet}



%Dokumentenanfang mit diversen input von mehreren Einzeldokumenten.
\begin{document}
	\begin{titlepage}
\center
\Large Ergänzungen zur Analysis 1 Sommersemester 2013\large \\[2em]
Dozent:\\Prof. Dr. Felix Leinen\\[2em]
Mitschrift von:\\Sven Bamberger, Maicon Hieronymus, Bernadette Mohr\\[2em]
\LaTeX{arbeit} von:\\Sven Bamberger, Maicon Hieronymus, Bernadette Mohr\\[2em]
Zuletzt Aktualisiert:\\\today\\
\includegraphics[scale=.2]{front/pics/Logo.jpg}\\\quad\\
\Large \textbf{Zusammenfassung:}\\[1em]
\parbox{0.75\textwidth}{\large
Es handelt sich hier um den Baustein "'Statistik und Ergänzungen zur Analysis"` des Pflichtmoduls "'Analysis und Statistik"`, der zu Beginn des Studienganges BSc Informatik im Kontext zum Baustein "'Analysis I"` absolviert werden soll.  Die "'Statistik und Ergänzungen zur Analysis"` wird nur in Sommersemestern angeboten.  Die Abschlußklausur zu "'Statistik und Ergänzungen zur Analysis"` ist zugleich Modulabschlußklausur des Pflichtmoduls "'Analysis und Statistik"`.\\
http://www.mathematik.uni-mainz.de/arbeitsgruppen/gruppentheorie/leinen/s13-ana-plus\\
\textbf{Raum:} Mi 03-428\\
\textbf{Uhrzeit:} 08:00-10:00\\
\textbf{Abgabe:} Mittwoch 08:13\\
}
\end{titlepage} 
	
	\frontmatter 
		\input{frontmatter/preface}
		\input{frontmatter/table_of_contents} 
	 
	\mainmatter 
		\chapter{Wahrscheinlichkeitsrechnung} 
 	
	\section{Binomialkoeffizienten}

%
%
%

\defi
\subsubsection{Fakultät}
$n! = n(n-1)(n-2)\dots1$\\
für $1\leq n\in\mathbb{N}$\\
$0!=1$\\

%
%
%

\stz
$n!$ ist die Anzahl der Möglichkeiten, $n$ Objekte anzuordnen (in einer Reihe)\\

%
%
%

\bws 

 \begin{tabular}{ll}
$n$  & Möglichkeiten für 1. Objekt \\
$(n-1)\dots$ & Möglichkeiten für 2. Objekt \\
$(n-2)\dots$ & Möglichkeiten für 3. Objekt\\
 \end{tabular}

$\square$
	\section{Grundbegriffe der Wahrscheinlichkeitsrechnung}

Viele Vorgänge in unserer Welt können wir nicht präzise auf ihre Ursachen zurückführen - sie erscheinen uns "`zufällig"'. Wir versuchen dennoch dieses Verhalten mathematisch exakt zu beschreiben. 
%
%
%
\defi 
Ein Zufallsversuch sei ein Experiment dessen mögliche Ausgänge als \underline{Ergebnisse} bezeichnet werden. Die Menge aller Ergebnisse sei die Ergebnismenge $\Omega$ des Versuchs.
%
%
%
\bsp
\begin{description}
	\item[$\blacktriangleright$] Werfen von zwei Würfeln $\Omega = S \times S$ wobei 
			$S=\{1,\cdots,6\}$
	\item[$\blacktriangleright$] Die täglich von 12 Uhr an einer Tafel gemessene Temperatur
			ist ein Ergebnis aus $\Omega = \mathbb{R}$
\end{description}
%
%
%
\defi 
Jede Teilmenge der Ergebnismenge $\Omega$ bezeichnen wir als ein \underline{Ergebnis}.\\
Wir sagen: Das Ereignis $A \subseteq \Omega$ \underline{tritt ein}, falls eine Durchführung des Zufallsversuchs ein Ergebnis liefert, das in $A$ liegt. 
%
%
%
\bsp
\begin{description}
	\item[$\blacktriangleright$] Werfen zweier Würfel:\\
			Pasch =$ \{ (1,1),(2,2),(3,3),\dotsc,(6,6)\} \subseteq \Omega = S \times S$ wie 
			oben.
	\item[$\blacktriangleright$] Hörsaaltemperatur: Angenehm = $ [22,24] \subseteq 
			\mathbb{R}$
\end{description}
%
%
%
\defi
Eine \underline{Ergebnisalgebra} in $\Omega$ sei eine Familie von Ereignissen mit: \begin{description}
	\item[$\blacktriangleright$] $\Omega \in$ \suet{A}
	\item[$\blacktriangleright$] \normalfont{Ist }$A\subset$ \suet{A} \normalfont{, so 
			auch }$\Omega \diagdown A \in$ \suet{A} \normalfont{}
	\item[$\blacktriangleright$] sind $A_{n} \in $\, \suet{A}\normalfont{} 		
			$(n\in\mathbb{N})$ so auch 
			$\mathop{\bigcup}\limits_{n\in\mathbb{N}}A_{n} \in$ \,\suet{A}
			\normalfont{}
			
\end{description}

Eine \underline{Wahrscheinlichkeitsfunktion} sei eine Abbildung $p$\, \suet{A}\normalfont{} $\rightarrow$ [0,1] mit:
\begin{description}
	\item[$\blacktriangleright$] $p(\Omega)=1$
	\item[$\blacktriangleright$] Sind $A_{n} \in$ \suet{A}\normalfont{}
		$(n\in\mathbb{N})$ mit $A_{n} \cap A_{m} =$ \O
\end{description}

Wir nennen $(\Omega,p)$ einen \underline{Wahrscheinlichkeitsraum}

%
%
%

\bsp 
\begin{description}
	\item[$\blacktriangleright$] Ist $\Omega$ endlich, so wähle \suet{A}\normalfont{} = 
		$\gamma ( \Omega)$= Menge aller Teilmengen von $\Omega$ und 
		$p(A)=\sum\limits_{w\in A} p(w)$ für $A \subseteq \Omega$ sofern $p=\Omega 
		\rightarrow$ [1,0] mit $\sum\limits_{w\in\Omega}p(w) = 1$
	\item[$\blacktriangleright$] $\Omega$ = Alphabet der deutschen Sprache \\
		$p(w)$ = Wahrscheinlichkeit für das Auftreten der Buchstaben $w$ in einem 
		Text der deutschen Sprache.
	\item[$\blacktriangleright$] \quad Fairer Münzwurf:\\
		$\Omega = \{z,k\} \qquad p(z) = \frac{1}{2} = p(k)$\\
		Gezinkter Münzwurf:\\
		$\Omega = \{z,k\} \qquad p(z) \neq \frac{1}{2} \neq p(k)$\\
		wobei: $p(z) = p$ und $p(k) = 1-p$
	\item[$\blacktriangleright$] \quad Werfen zweier fairer Würfel:\\
		$\Omega = S\times S \qquad p(w) = \frac{1}{36}$ für alle $w \in \Omega$ wie 
		oben.
\end{description}	

%
%
%
\defi
Ist $|\Omega| < \infty$ und $p(w) = \frac{1}{|\Omega|}$ für alle $w \in \Omega$, so nennen wir $p$ \underline{Gleichverteilung} auf $\Omega$

%
%
%

\stz
Für Ereignisse $A,B \in $ \suet{A}\normalfont{} gilt stets:
\begin{enumerate}[a) ]
	\item $A\cap B \in$ \suet{A}\normalfont{}
	\item $p(\Omega \diagdown A)=1-p(A)$, insbesondere $p($\O$) = 0$
	\item ist $A \subseteq B$, so ist $p(A) \leq p(B)$
	\item $p(A\cup B) = p(A) + p(B) - p(A\cap B)$
\end{enumerate}
\qquad \\
\qquad\\
%
%
%

\bws \\
\begin{enumerate}[a) ]
	\item $\Omega\diagdown(A\cap B) \mathop{=}\limits^{\textrm{DE MORGAN}} 
		(\Omega \diagdown A) \cup (\Omega \diagdown B)$
	\item $\Omega = A \cup ( \Omega \diagdown A)$ \\
		$1=p(\Omega) = p(A) + p(\Omega \diagdown A)$
	\item $B=A\dot{\cup}(B\diagdown A)$ mit $B\diagdown A = B \cap (\Omega 
		\diagdown A) \in$ \suet{A} \normalfont{}
	\item $(A\cup B)\diagdown A = B \diagdown (A\cap B)$\\
		Ansonsten gab es nur Fehler bei dem erläuterungsversuch in dieser Vorlesung.
\end{enumerate}














%		\input{mainmatter/chapter_2}
	 
	\backmatter 
		\input{backmatter/listoffigures} 
\end {document}